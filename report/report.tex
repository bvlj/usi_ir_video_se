\documentclass[12pt]{exam}

\usepackage{amsmath}
\usepackage{enumitem}
\usepackage[margin=2.25cm]{geometry}
\usepackage{graphicx}
\usepackage[bookmarksopen, bookmarksdepth=2]{hyperref}
\usepackage{listings}
\usepackage{tikz}
\usepackage{titling}
\usepackage{xcolor}

\newcommand{\lp}{\left(}
\newcommand{\rp}{\right)}

\newcommand{\n}[0]{
\hspace*{\fill}
}

\newcommand{\todo}[1]{{
\color{red} \textbf{TODO}: {#1} \n
}}

\newcommand{\smallpic}[1]{
\includegraphics[width=0.4\textwidth]{img/#1}
}

\newcommand{\pic}[2]{{
\begin{center}
\includegraphics[width=0.6\textwidth]{img/#1} \\
{#2}
\end{center}
}}

\lstset{
basicstyle=\ttfamily\footnotesize,
breaklines=true
}
\hypersetup{
colorlinks=true,
linkcolor=black,
filecolor=cyan,
urlcolor=blue,
}

\makeindex

\setlength{\droptitle}{-5em}

\title{
{\protect\includegraphics[width=\textwidth]{logo-info.pdf}}
\protect\textbf{Information Retrieval Course Project Report}\n
\protect Video Search Engine
}

\author{Bevilacqua Joey}

\setlength{\parindent}{0cm}
\pagestyle{plain}

\begin{document}

\maketitle

\tableofcontents

\newpage

\section{Introduction}

\subsection{Requirements}

The Information Retrieval Course project required to build a search engine
that allows users to search for videos by submitting some keywords. 
The search engine shall provide the user an interface for searching, browsing
and presenting the results. One last requirement that was given for this
project is that the results of the search engine would be given using
Pseudo Relevance Feedback.

\subsection{The challenges}

This project turned out challenging due to the different sources the
indexed videos come from: websites that were crawled are built with different
technologies and required the development of different approaches to extract
the information that was needed.

Providing a good user interface was also challenging: videos are not simply
representable without lots of graphical content that can become overwhelming
pretty fast, especially when a lot of results are presented, the right amount
of graphics and textual information had to be correctly placed in order to
keep the user comfortable and make the search experience enjoyable.

\subsection{The result}

The search engine that was built is called \textit{Ok Video}: it allows
users to easily discover new videos from different sources with a simple
focused user interface.

\pic{se_home}{The \textit{Ok Video} home page}

\newpage

\section{Crawling \& Indexing}

To find videos and make them available to users using the search engine
three powerful tools were used: \textit{Scrapy}, \textit{Selenium}
and \textit{Solr}.

The first two were used to crawl for videos on various sources while
\textit{Solr} was used to index data and provide a ``search API''.

\subsection{Crawling}

The following websites were crawled to fetch videos for the search
engine: \href{https://talksat.withgoogle.com}{Talks at Google},
\href{https://www.ted.com}{Ted.com}, \href{https://www.thersa.org}{The RSA} and
\href{https://vimeo.com}{Vimeo}. \n

To crawl \textit{Talks at Google}, \textit{Ted.com} and \textit{The RSA}
\textit{Scrapy} was used. \textit{Scrapy} is a simple and fast framework to
build web spiders and can process multiple webpages synchronously.
\textit{Vimeo}, on the other hand did not work with \textit{Scrapy} due to its
reliance on \textit{JavaScript} code to present content, while \textit{Scrapy}
has no \textit{JavaScript} engine to process said code. \n

To solve this problem \textit{Selenium WebDriver} was used instead:
\textit{Selenium WebDriver} is a powerful web browser automation tool,
often used for website testing and automation-guided exploration. This powerful
tool was used to take control of a headless Firefox instance and use it to
visit (and crawl) \textit{Vimeo}. This allowed to let the \textit{Vimeo}
website make use of its \textit{JavaScript} code while still allowing us to
have a \textit{Scrapy}-like tool to crawl and extract information from it.

The crawling of each website has been automated as a python script that
produced \texttt{json} files with the crawled data.

\subsection{Indexing}

\textit{Apache Solr} is a popular, powerful and open source search platform
that allows to easily create a search system that has powerful matching
capabilities including phrases, wildcards, joins, grouping and much more
across any data type. \n

The \texttt{json} outputs of the web crawlers were the posted to
an appositely created \textit{Solr collection} so that they could be served
as search results in the search engine.

The collection was specified to have this schema (in addition to the default
configuration):

\begin{center}
\begin{tabular}{lll}
Field name & Type          & Description                \\\hline
author     & text\_general & Video author               \\
category   & string        & Video category / topic     \\
image      & string        & Video preview image url    \\
source     & string        & Video website source       \\
title      & text\_en      & Video title                \\
url        & string        & Video url                  \\\hline
\_text\_   & text\_general & Copy field used for search \\
\end{tabular}
\end{center}

\newpage

\section{The user interface}

The search engine has a graphical user interface accessible from a web
browser as one would expect from an usual search engine that provides
access to other resources on the web.

The user interface is written in TypeScript and uses React, a JavaScript
framework that allows declarative user interfaces.

\subsection{The homepage}

\pic{se_home}{The HomePage}

The homepage is composed of a limited number of elements: at the very
top we have the website title, a search bar and some \textit{Channels}.

\subsection{Searching}

\pic{se_query}{The results of the query \textit{Quantum computers}}

By inserting a query in the search bar we can search for videos.
The results are presented using the pseudo-relevance-feedback technique:
first the user query is sent to Solr, which will then return five (5) 
results for the original user query. The system then reads the five (5)
titles of the results and computes the td-idf for each word, and then we take
the top five (5) words and add it to the original query.

The formula that was used to perform the pseudo-relevance-feedback is the
following:

$$
tfidf \lp t, d, C \rp = tf \lp t, d \rp \times idf \lp t, C \rp
$$
$$
tf = \dfrac{N}{1 + \left\{ d \in C : t \in d \right\}}
$$
$$
idf = log \lp \dfrac{N}{1 + \left\{ d \in C : t \in d \right\}} \rp
$$

These keywords are then sent back to Solr which will return the complete list
of videos that will be shown to the user as the result of the query.

Implementation of the pseudo-relevance-feedback \textit{algorithm} can be
found in this file: \texttt{ui/src/util/prf.ts}.

\subsection{Channels}

\pic{se_channel}{The \textit{food} channel (\texttt{f})}

Channels provide a simple way to access collections of videos pertaining
to certain topics. Channels are system-generated and users cannot add
their own channels, even if it'd be an interesting feature to develop if
the search engine were to be expanded.

Channels can be easily queried for using a shortcut character in the search
bar or by clicking on them in the home page.

\subsection*{Channel shortcuts}

By inserting the shortcut character in the search bar, will show the channel
rather than results of the query for that character.

\begin{center}
\begin{tabular}{ll|ll|ll}
Channel & Shortcut & Channel & Shortcut & Channel & Shortcut \\\hline
Art               & a & Economy           & e & Food               & f \\
Health \& Fitness & h & Literature        & l & News \& Journalism & j \\
Music             & m & Animals \& Nature & n & Sports             & o \\
Politics          & p & Science           & s & Technology         & g \\
\end{tabular}
\end{center}

\subsection{Filtering}

\pic{se_filter}{
Results for the query \textit{geopolitics}, filtered for \textit{Gilles Kepel}
}

It is possible to filter the search results using the filter bar or by
appending a slash ( \texttt{/} ) after the user query. The filter will
then allow you to shown only results containing the filter string in the
title, author name or topic name.

Searching for \texttt{geopolitics/Gilles Kepel} will yield the same result as
searching for \textit{geopolitics} and then inserting \textit{Gilles Kepel} in
the filter bar.

Filtering works on Channels too, so searching for \texttt{a/paintings} will
show the Art channel with the filter \textit{paintings} applied.

\subsection{Advanced features}

The search engine's user interface was built to be pretty simple and clean,
but there exist more features for expert users who may want to tune their
search results to obtain the information that they need faster and more
efficiently.

\subsubsection{Results exclusion}

\begin{center}
\begin{tabular}{cc}
\smallpic{se_search} & \smallpic{se_exclude}
\end{tabular} \\
Search results are different when we exclude the word ``Chris''
(name of author of the first video in the results on the left).
\end{center}

By pre-pending a dash ( \texttt{-} ) to a word, we can explicitly exclude
it from the search results. This exclusion is applied after the 
pseudo-relevance-feedback process so that we ensure final results do not
include what the user is not looking for.

\subsubsection{Results sorting}

\pic{se_sorting}{Search results are sorted by Author in alphabetic order}

It's possible to sort videos by title, author or topic in alphabetic 
or reverse alphabetic order by clicking on the title of the column.

\subsection{OnBoarding}

\pic{se_onboarding}{
The \textit{Teaching popup} explains the filter bar to the user
}

While most of the user interface is self-explanatory, more advanced features
might be harder to discover without seeing someone else using them or
reading about them in a documentation.

That's why there exists a \textit{Teaching popup} that will appear at the
right time to explain how users can improve their searching experience with
quick tips and concrete examples.

\subsection{404 - Not found}

\pic{se_fourzerofour}{The \textit{404} page}

When trying to access a link that's not available (anymore), the
user will be presented a friendly user interface with inspiration
for some channels to watch a video from.

\newpage

\section{User evaluations}

The user evaluation consisted in three (3) different users
testing the search engine on their own machine in a comfortable and
familiar environment. Users were asked to answer some background-check
questions, perform a number of predefined task and finally to let them play
around with the system by performing any number of queries they'd have liked.

This approach allowed to perform a more complete evaluation of the system by
having a bit of ``operational evaluation'' with the limitation that users could
not use their own collection.

Here are the predefined tasks users were asked to perform for the evaluation:

\begin{enumerate}
\item Find videos about "Quantum Computing"
\item Find videos about "Internet trolls" that contain the word "inside" 
      in their title
\item Find videos about "Geopolitics" and sort them by author name
\end{enumerate}
\subsection{User A}

User A has good background knowledge of search engines and is also used to
searching videos on the internet using various search engines and video hosting
platforms.

User A appreciated the style of the user interface but, even if the
filter text box was found useful, the slash feature in the search bar for
automatic filtering was found a bit hard to grasp at the beginning. Channels
were seen as an interesting idea that could be expanded with even more options
and contents.

User A suggested to implement another \textit{Teaching bubble} (which was
much appreciated for its style and interesting tips) for sorting videos, as it
was said that it might not be clear to more less experienced users that it is
possible to sort by clicking on the results headers.

User A finally suggested to add a ``back arrow'' / ``home button'' to let
users easily go back to the main homepage of the search engine.

\subsection{User B}

User B has good background knowledge of search engines and is also used to
searching videos on the internet using various search engines and video hosting
platforms.

User B appreciated very much the user interface but found that some results
were not ``literally matching'' the query that was given to the system. This
happened due to the pseudo-relevance-feedback that modified the query as it was
expected to do: nonetheless this is a signal that the pseudo-relevance-feedback
implementation is not perfect, which is understandable due to the usage of a
prebuilt general-purpose NLP model.

User B did not make use of filters even when hinted to do so in tasks. Channels
were also ignored for the majority of the evaluation, only being used at the
very end instead of performing a simple query, but they made a good impression
thanks to satisfactory results.

\subsection{User C}

User C has good background knowledge of search engines and is also used to
searching videos on the internet using various search engines and video hosting
platforms.

User C appreciated the user interface and when User C read the onboarding
\textit{Teaching popup}s, immediately fell in love with filters and Channels
functionality which allowed User C to perform tasks by writing better queries.
User C said that this feature was conceptually similar to the
DuckDuckGo's Bangs.

User C suggested to expand the filters and channels functionalities by 
integrating the ``topic'' field of videos and providing ways to group even more
contents in channels.

\subsection{Conclusion}

The three (3) user evaluation highlighted different strengths and weaknesses
of this implementation, providing many insights for future improvements:

\begin{itemize}
\item More teaching bubbles to help more \textit{novice} users know about
      advanced features.
\item Clearer ``back home'' navigation.
\item Improvements to the quality of the pseudo-relevance-feedback
      implementation.
\item Expansion of the Channels to include even more (if not all) topics
      for a better organized search experience for advanced users.
\end{itemize}

\end{document}